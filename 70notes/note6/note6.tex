\documentclass[a4paper]{article}

\usepackage[english]{babel}
\usepackage[utf8]{inputenc}
\usepackage{amsmath}
\usepackage{graphicx}
\usepackage{amssymb}
\usepackage{amsthm}
\usepackage{tikz-cd}
\usepackage{mathrsfs}
\usepackage[colorinlistoftodos]{todonotes}
\usepackage{enumitem}
\usepackage{yfonts}
\usepackage{ dsfont }
\usepackage{array}
\usepackage{mathabx}
\usepackage{tabu}  
\usepackage{color,soul}

\title{CS70 In Simpler Terms - Note 6}
\author{written by Kevin Liu cs70.kevinmliu.com}
\date{July 21, 2017}
\newtheorem{thm}{Theorem}[section]
\newtheorem{lem}[thm]{Lemma}

\newtheorem{defn}[thm]{Definition}
\newtheorem{eg}[thm]{Example}
\newtheorem{ex}[thm]{Exercise}
\newtheorem{conj}[thm]{Conjecture}
\newtheorem{cor}[thm]{Corollary}
\newtheorem{claim}[thm]{Claim}
\newtheorem{rmk}[thm]{Remark}

\newcommand{\ie}{\emph{i.e.} }
\newcommand{\cf}{\emph{cf.} }
\newcommand{\into}{\hookrightarrow}
\newcommand{\dirac}{\slashed{\partial}}
\newcommand{\R}{\mathbb{R}}
\newcommand{\C}{\mathbb{C}}
\newcommand{\Z}{\mathbb{Z}}
\newcommand{\N}{\mathbb{N}}
\newcommand{\Q}{\mathbb{Q}}
\newcommand{\LieT}{\mathfrak{t}}
\newcommand{\T}{\mathbb{T}}
\newcommand{\A}{\mathds{A}}

\begin{document}
\maketitle

\section{Random Variables and Expectations}
There are a few main distributions that we will be focusing on in this course, but understanding the probability aspect and how to calculate their Expectations will be key to your success.\\\\
\textbf{Random Variabe:} a function $X: \omega \Rightarrow R$ that assigns a real number to every outcome $\omega$ in the probability space. ($\omega$ is a sample point) \\
\indent \textit{Example: $X(\omega) = number of heads$, then $X(THH) = 2$}\\\\
\textbf{Expectation and Linearity:}
\begin{itemize}
    \item Expectation is the mean/average of the r.v.: $E(x)= \sum_a xP(x=a)$
    \item Linearity:
    \begin{itemize}
        \item For $2$ r.v. $x,y$ on the same probability space: $E(X+Y) = E(X) + E(Y)$
        \item For any constant $c$, $E(cX) = cE(X)$
        \item \textbf{Note: Linearity of Expectation holds even when the r.v. is NOT Independent}
    \end{itemize}
    \item $E(XY) = E(X)E(Y)$ \textit{iff} $X,Y$ are independent
    \item The expected number of fixed points in a random permutation of $n$ items is always $1$ regardless of $n$ (e.g. $E(X) = 1$)
    \item Indicator Variables general form:\\
    \begin{itemize}
        \item Let $X_i$ be the indicator of of whether...
        \item $X = X_1 + X_2 + ... X_n \Rightarrow E(X) = \sum_{i=1}^n E(X_i)$
        \item $E(X^2) = E([X_1 + X_2 + .. X_n]^2) = \sum_{i=1}^n E(X_i)^2 + \sum_{i\neq j}^{n(n-1)}E(X_i X_j)$ \textbf{Explanation:} \textit{there are $n(n-1)$ unlike terms in the expansion of $(X_1 + X_2 + .. X_n)(X_1 + X_2 + .. X_n)$(after selecting $1$ of the $n$ terms in the first set, you have $n-1$ other terms to choose from in the second of the pair)}
        \item $E(X_i) = E(X_i^2)$ - since you only care about $Pr(X_i=1) \Rightarrow \\E(X_i) = 1 * Pr(X_i = 1) + 0 * Pr(X_i = 0)$
        
    \end{itemize}

\end{itemize}

\section{Variance and Covariance}
\textbf{Definition:} $Var(x) = E((x-\mu)^2) = \sigma^2$, where $\sigma$ is standard deviation. So $Var(x) = (\sqrt\sigma)$
\begin{itemize}
    \item For r.v. with $E(X) = \mu$, $Var(X) = E(X^2) - \mu^2 = E(X^2) - E(X)^2$
        \begin{itemize}
            \item Proof: $Var(X) = E((x-\mu)^2) = E(X^2 - 2\mu X + \mu^2)$\\ $= E(X^2) - 2\mu E(x) + \mu^2$ and substituting $\mu for E(X)$, we get \\ $Var(X) = E(X^2) - \mu^2 = E(X^2) - E(X)^2$
        \end{itemize}
    \item $Var(cX) = c^2Var(X)$
    \item Independent r.v.: $P(X=a, Y=b) = P(X=a)P(Y=b)$ $\forall a,b$
    \item Covariance
    \begin{itemize}
        \item Covariance of $X,Y = E(XY) - E(X)(Y)$, and if $X,Y$ are independent, the covariance is $0$
        \item $Var(X+Y) = Var(X) + Var(Y) + 2Cov(X,Y)$ and $Cov(X,Y) = 0$ if $X,Y$ are independent
        \item $Cov(a + bX, c + dY) = bdCov(X,Y)$
        \item $Cov(X_1 + X_2, Y) = Cov(X_1,Y) + Cov(X_2,Y)$
    \end{itemize}

    \item Note that the value of $Var(X) \geq 0$ so $E(X^2) \geq E(X)^2$

\end{itemize}




\end{document}
