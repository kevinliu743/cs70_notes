\documentclass[a4paper]{article}

\usepackage[english]{babel}
\usepackage[utf8]{inputenc}
\usepackage{amsmath}
\usepackage{graphicx}
\usepackage{amssymb}
\usepackage{amsthm}
\usepackage{tikz-cd}
\usepackage{mathrsfs}
\usepackage[colorinlistoftodos]{todonotes}
\usepackage{enumitem}
\usepackage{yfonts}
\usepackage{ dsfont }
\usepackage{array}
\usepackage{mathabx}
\usepackage{tabu}  
\usepackage{color,soul}

\title{CS70 In Simpler Terms - Note 11}
\author{written by Kevin Liu cs70.kevinmliu.com}
\date{August 2, 2017}
\newtheorem{thm}{Theorem}[section]
\newtheorem{lem}[thm]{Lemma}

\newtheorem{defn}[thm]{Definition}
\newtheorem{eg}[thm]{Example}
\newtheorem{ex}[thm]{Exercise}
\newtheorem{conj}[thm]{Conjecture}
\newtheorem{cor}[thm]{Corollary}
\newtheorem{claim}[thm]{Claim}
\newtheorem{rmk}[thm]{Remark}

\newcommand{\ie}{\emph{i.e.} }
\newcommand{\cf}{\emph{cf.} }
\newcommand{\into}{\hookrightarrow}
\newcommand{\dirac}{\slashed{\partial}}
\newcommand{\R}{\mathbb{R}}
\newcommand{\C}{\mathbb{C}}
\newcommand{\Z}{\mathbb{Z}}
\newcommand{\N}{\mathbb{N}}
\newcommand{\Q}{\mathbb{Q}}
\newcommand{\LieT}{\mathfrak{t}}
\newcommand{\T}{\mathbb{T}}
\newcommand{\A}{\mathds{A}}

\begin{document}
\maketitle

\section{Error Correcting Codes}
Suppose you are trying to send a message of $n$ packets. There are two cases:
\begin{itemize}
    \item Erasure code: $k$ packets are erased, so you need at least $n+k$ packets 
    \item General error: $k$ packets are corrupted, so you need at least $n + 2k$ packets \\
    \textit{This can be thought about intuitively in that if there are $k$ corrupt packets, then you are left with $2k$ good packets, and the $2k$ good packets match up so you are more confident with your decision on which packets are corrupt}\\
    Suppose you are trying to reconstruct $P(x)$ from the $(n+2k)^+$ packets
    \begin{itemize}
        \item (good) $P(i) = r(i)$ (received) for at least $n+k$ packets
        \item let $E(x) = (x-e_1)...(x-e_k)$, where $e_i =$ a corrupted packet, and there are $k$ of those
        \item $P(i)E(i) = r_i E(i)$ for $1 \leq i \leq n+2k$ (all packets)\\
        since $P(i) = r_i$ for good packets and $E(i) = 0$ for corrupt packets
        \item We now need to show that this is solvable:
        \begin{itemize}
            \item Let $Q(x) = P(x)E(x)$, then $P(x) = \frac{Q(x)}{E(x)}$
            \item $P(x)$ is of degree $n-1$ (\textit{$n$ packets are necessary to construct an $n-1$ degree polynomial}), $E(x)$ is of degree $k$, so $Q(x)$ is of degree $n+k-1$, 
            \item $Q(x)= a_{n+k-1}x^{n+k-1} + ...a_1x + a_0$ and there are $n+k$ unknown coefficients
            \item $E(x) = 1x^k + (b_{k-1}x^{k-1} + ...b_1x + b_0)$ and there are $k$ unknown coefficients
            \item There are a total of $n+2k$ unknown coefficients, and since we have $n+2k$ total packets, we can contruct $n+2k$ equations, and when \textit{the number of equations = number of variables}, it's solvable!
            \item \textbf{Note:} If the question states that the polynomial is restricted to a finite field, remember to $\pmod q$ constantly, where $q$ is the field you are working within\\
            \textit{Example:}
            \begin{itemize}
                \item let $E(x) = x + b_0$
                \item if $k=1, b_0= 6,$ and the field $q = 7\\ \Rightarrow E(x) = (x + 6) \pmod7 \equiv (x - 1)\pmod7$, so $e_1 = 1$\\
                Your solution to $E(x) = x - 1$
            \end{itemize}
        \end{itemize}
    \end{itemize}
\end{itemize}


\section{Secret Sharing}
\textit{Example:}
\begin{itemize}
        \item any group of $k$ can reconstruct the polynomial, $P(x)$ to figure out the secret code (usually at $P(0)$)
        \item If the group is $k-1$ or fewer, then they can't construct the polynomial easily \textit{it will later be shown that having a group of $k-1$ people gives you no more information about the secret code than being given no values at all.}
        \item Let $q$ be a very large prime number $>> n,s$
        \item $P(x)$ is of degree $k-1$ and $P(0) = $ s (secret code)
        \item $P(x) = a_1x^n + a_2x^{n-1}+...a_n + s$
        \item $P(1)$ is given to the first person, $P(2)$ is given to the second person, ... $P(n)$ is given to the $n$th person 
        \item If there are only $k-1$ values, $s$ can take on the values $\{0,1,... q-1\} = q$ values still (as described in the chart from the previous note). And since you are working over $\pmod q$, there were $q$ possibilities to begin with! so you had no more information with $k-1$ values than with $0$ values...
\end{itemize}

\section{More Examples}
\begin{itemize}
    \item $P(x)$ degree $4$, $Q(x)$ is degree 2. What is the maximum number of intersections of the two polynomials?
    \begin{itemize}
        \item $P(x) - Q(x) = 0$ has at most 4 zeroes = 4 points of intersections
    \end{itemize}
    \item Simplify (remove the denominator) for the following:\\  $\frac{(x-4)(x-5)}{2} \pmod 7 \equiv (x-4)(x-5)(\frac{1\pmod 7}{2\pmod 7})$
    \begin{itemize}
        \item $\frac{1\pmod 7}{2\pmod 7} \equiv \frac{8}{2} \pmod 7 \equiv 4\pmod 7$ OR
        \item $\frac{1\pmod 7}{2\pmod 7} = x\pmod 7 \Rightarrow 1 \equiv 2x\pmod 7 \Rightarrow 2^{-1} = 4 \equiv 4\pmod 7$
    \end{itemize}
\end{itemize}




\end{document}
