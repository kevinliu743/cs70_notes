\documentclass[a4paper]{article}

\usepackage[english]{babel}
\usepackage[utf8]{inputenc}
\usepackage{amsmath}
\usepackage{graphicx}
\usepackage{amssymb}
\usepackage{amsthm}
\usepackage{tikz-cd}
\usepackage{mathrsfs}
\usepackage[colorinlistoftodos]{todonotes}
\usepackage{enumitem}
\usepackage{yfonts}
\usepackage{ dsfont }
\usepackage{array}
\usepackage{mathabx}
\usepackage{tabu}  
\usepackage{color,soul}

\title{CS70 In Simpler Terms - Note 3}
\author{written by Kevin Liu cs70.kevinmliu.com}
\date{July 2, 2017}
\newtheorem{thm}{Theorem}[section]
\newtheorem{lem}[thm]{Lemma}

\newtheorem{defn}[thm]{Definition}
\newtheorem{eg}[thm]{Example}
\newtheorem{ex}[thm]{Exercise}
\newtheorem{conj}[thm]{Conjecture}
\newtheorem{cor}[thm]{Corollary}
\newtheorem{claim}[thm]{Claim}
\newtheorem{rmk}[thm]{Remark}

\newcommand{\ie}{\emph{i.e.} }
\newcommand{\cf}{\emph{cf.} }
\newcommand{\into}{\hookrightarrow}
\newcommand{\dirac}{\slashed{\partial}}
\newcommand{\R}{\mathbb{R}}
\newcommand{\C}{\mathbb{C}}
\newcommand{\Z}{\mathbb{Z}}
\newcommand{\N}{\mathbb{N}}
\newcommand{\Q}{\mathbb{Q}}
\newcommand{\LieT}{\mathfrak{t}}
\newcommand{\T}{\mathbb{T}}
\newcommand{\A}{\mathds{A}}

\begin{document}
\maketitle

\section{Infinity and Countability}
Countability may be a fairly new concept for many of you, and it is easy to confuse computability and countability. Here I will summarize what you need to know as well as a few tips that will guarantee you a few basic points on the exam. 
\begin{itemize}
    \item $2$ sets have the same \textit{cardinality}/size if we can demonstrate a bijection between the two sets(show onto and one-to-one)
    \item Set $S$ is \textit{countable} if there is a bijection between $S$ and $N$
    \item $|\N| = |\Z| = |\Q|$ (all countable sets)
    \item if $|A| \leq |B|$ and $|B| \leq |A| \Rightarrow |A| = |B|$
    \item Binary strings of any finite length: $\{0,1\}^*$ (each digit in a binary string is from the set \{0,1\})
    \item Ternary strings: $\{0,1,2\}^*$ 
    \item Lexicographic order: numerically increasing order
    \item Cantor's Diagonalization proof: proves that $\R$ is not countable by adding $2\pmod{10}$ to each of the values in the diagonal, and noticing that the diagonal number can't exist in the set. This proof can't be used on $\Q$ since adding $2\pmod{10}$ to a rational number does not guarantee that it wil still be a rational number.    

\end{itemize}

\section{Computability and the Halting Problem}
In this section, I will highlight a few of the most common examples when dealing with computability problems. All of these examples rely on the fact that the Halting problem program does not exist. 
\textbf{General Halting Problem approach:} \\\\
Assume by contradiction that program $P$ exists.\\
\indent define Halt \\
\indent Modify $F \Rightarrow F^{'}(x)$\\
\indent Use $P$ as a subroutine\\
\indent If the original program halts, $P$ returns true, otherwise false\\

\textit{Example problem: \\ \indent Consider a program $P$ that takes in $F$, input $x$, output $y$,\\ \indent returns true if $F(x)$ outputs $y$, and returns false otherwise. 
}\\
\indent def Halt(F, x):\\
\indent \indent $y = 0$ \\
\indent \indent def $F^{'}(x):$\\
\indent \indent \indent $F(x)$ \\
\indent \indent \indent return y \\
\indent \indent return $P(F^{'}, x, y)$\\



\begin{itemize}
    \item Can a computer program print all rational numbers? \\ \indent Yes, since you can enumerate $\Q$ so you can print them.
    \item There is NO program DEAD which takes $P, x, n$ and determines if the $nth$ line is executed when you run $P(x)$. 
    \item There exists a program $H$ that determines whether a program $P$ on input $x$ that outputs the value $x + 42$ after executing $42$ statements or \textit{steps}. True statement. 
    \item \textbf{**IMPORTANT**} You can count the number of steps that a Program has taken, but you can't determine whether a line has been executed. 
\end{itemize}



\end{document}
