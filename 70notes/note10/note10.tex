\documentclass[a4paper]{article}

\usepackage[english]{babel}
\usepackage[utf8]{inputenc}
\usepackage{amsmath}
\usepackage{graphicx}
\usepackage{amssymb}
\usepackage{amsthm}
\usepackage{tikz-cd}
\usepackage{mathrsfs}
\usepackage[colorinlistoftodos]{todonotes}
\usepackage{enumitem}
\usepackage{yfonts}
\usepackage{ dsfont }
\usepackage{array}
\usepackage{mathabx}
\usepackage{tabu}  
\usepackage{color,soul}

\title{CS70 In Simpler Terms - Note 10}
\author{written by Kevin Liu cs70.kevinmliu.com}
\date{August 1, 2017}
\newtheorem{thm}{Theorem}[section]
\newtheorem{lem}[thm]{Lemma}

\newtheorem{defn}[thm]{Definition}
\newtheorem{eg}[thm]{Example}
\newtheorem{ex}[thm]{Exercise}
\newtheorem{conj}[thm]{Conjecture}
\newtheorem{cor}[thm]{Corollary}
\newtheorem{claim}[thm]{Claim}
\newtheorem{rmk}[thm]{Remark}

\newcommand{\ie}{\emph{i.e.} }
\newcommand{\cf}{\emph{cf.} }
\newcommand{\into}{\hookrightarrow}
\newcommand{\dirac}{\slashed{\partial}}
\newcommand{\R}{\mathbb{R}}
\newcommand{\C}{\mathbb{C}}
\newcommand{\Z}{\mathbb{Z}}
\newcommand{\N}{\mathbb{N}}
\newcommand{\Q}{\mathbb{Q}}
\newcommand{\LieT}{\mathfrak{t}}
\newcommand{\T}{\mathbb{T}}
\newcommand{\A}{\mathds{A}}

\begin{document}
\maketitle

\section{RSA}
\begin{itemize}
    \item Encrypt message x: $E(X) = x^e\pmod N$, $N = pq$ (large primes)
    \item Decode $m = E(x)$: $D(m) = m^d \pmod N$, where $d$ is the M.I. of $e\pmod (p-1)(q-1) \Rightarrow gcd(e, (p-1)(q-1)) = 1$ ensures that $d$ exists \textit{since $e$ is relatively prime with $(p-q)(q-1)$}
    \item $D(E(x)) = x\pmod N \Rightarrow E$ is a bijection - \textbf{$x^{ed} = x\pmod N$} \newline Proof:
    \begin{itemize}
        \item $ed = 1\pmod (p-1)(q-1) = 1 + (p-1)(q-1)k$
        \item $x^{ed} - x = x^{1+k(p-1)(q-1)} - x = x(x^{k(p-1)(q-1)} - 1)$ \newline We are trying to show that the above statement $\equiv 0 \pmod N$, because then we will have proven that $x^{ed} \equiv x \pmod N$
        \item We can do this by showing that $x$ is divisibly by $p$ and $q$ $\Rightarrow$ divisible by $pq$
        \item case 1: $p | x$ \newline assuming that $x \neq 0\pmod p$ and using FLT \newline
        $x^{p-1} \equiv 1\pmod p \Rightarrow (x^{p-1})^{k(q-1)} \equiv 1^{k(q-1)} \pmod p \equiv 1 \pmod p \Rightarrow x^{k(p-1)(q-1)} -1 \equiv 0 \pmod p \qed$
        \item the proof for the case where $q|x$ is analagous
     \end{itemize}
\end{itemize}
\section{Polynomials}
\begin{itemize}
        \item degree $d$ polynomial has at most $d$ roots
        \item need $d+1$ points to uniquely determine polynomial of degree $d$
        \item Lagrange interpolation - used to find a polynomial of lowest degree, given a set of points \\\\
        \textit{Formula for calculating the final polynomial $p(x)$ of degree $d$} \newline $p(x) = \sum_{i=1}^{d+1}y_i \Delta_i(x)$, after constructing $d+1$ polynomials $\Delta_i(x)$\\\\
        \textit{Formula for finding the individual polynomials, $\Delta_i(x) = \frac{\Pi_{j \neq i}(x - x_j)}{\Pi_{j \neq i}(x_i - x_j)}$} \\\\\\\\
        \textit{Example:}
        Find the polynomial of lowest degree that passes through the points $(1,1), (2,2), (3,4)$
        \begin{itemize}
            \item $d = 2, x_i = i$
            \item $i = 1$, $\Delta_1(x) = \frac{(x-2)(x - 3)}{(1-2)(1-3)}$ 
            \begin{itemize}
                \item This makes sense because at $x = 2, x= 3,$  $\Delta_1(x) = 0$, and at $x=1,$ $\Delta_1(x) = 1$, so $y_1\Delta(x) = 1(1) = 1$ which is correct
            \end{itemize}
            \item $i = 2$, $\Delta_2(x) = \frac{(x-1)(x - 3)}{(2-1)(2-3)}$
            \item $i = 3$, $\Delta_3(x) = \frac{(x-1)(x - 2)}{(3-1)(3-2)}$
        \end{itemize}
\end{itemize}

\section{Finite Fields $F_m$ or $GF(m)$}
Restricting the fields to$\pmod m$
\begin{itemize}
    \item $x$ has inverse$\pmod m$ iff$gcd(m,x) = 1$ so $m$ is prime
    \item $x \in \{1,... m-1\}$ all have an inverse$\pmod m \Rightarrow $ bijection
    \item How many polynomials are there of degree at most $2 \pmod m$?
    \begin{itemize}
        \item There are $3$ coefficients in a degree 2 polynomial
        \item Each coefficient has $m$ values to choose from $\{0, ...m-1\} \Rightarrow m^3$ polynomials
    \end{itemize}
\end{itemize}
The chart below summarizes the result above:\\\\
\centering
\textbf{Polynomials of degree $\leq d$ over $F_m$}
    \begin{table}[htbp]
    \centering
    \begin{tabular}[t]{|c|c|}
        \hline
        Number of points given & Number of polynomials\\ \hline
        $d+1$ & 1\\ \hline
        $d$ & $m$\\ \hline
        . & .\\ 
        . & .\\
        . & .\\ \hline
        0 & $m^{d+1}$\\ \hline
    \end{tabular}
    \end{table}



\end{document}
